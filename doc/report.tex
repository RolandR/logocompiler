\documentclass[a4paper, 12pt, parskip=full*]{scrreprt}


% --------------PACKAGES----------------
% INPUT. FONT ENCODING & LANGUAGE
\usepackage[utf8]{inputenc}
\usepackage[T1]{fontenc}
\usepackage[ngerman]{babel}
%\usepackage[english]{babel}
\usepackage{graphicx}
\usepackage{natbib}
\usepackage{blindtext}
\usepackage{listings} 
\lstset{language=logo} 




% FIX SPACING BETWEEN LIST ITEMS
\usepackage{enumitem}
\setlist[itemize]{parsep=0pt}
\setlist[enumerate]{parsep=0pt}




%-----------TITELSEITE--------------
\title{Logo Project}
\subject{
	Bern University of Applied Sciences\\
	\vspace{1cm}
	Automata and Formal Languages \\
	Semester Project \\
	Bern \\ BTI7064r
}
\author{
    Bolaños Mayta Raul Alberto \\ bolar1@bfh.ch\and
    Roland Rytz \\ roland.rytz@gmail.com
}
\date{\today}

\begin{document}

\maketitle

%---------INHALTSVERZEICHNIS-------------
\tableofcontents

\chapter{Introduction} 
Logo is a easy to lean not case sensitive programing language in which his commands represents movements of a virtual turtle that  produces lines graphics.

In this Project we develop a \texttt{parser / translator} of the Logo programming language into Java. 
This logo translator should be able to run all the Logo programs provides as examples without modifications. 
Logo 

\chapter{Grammar}
\textbf{For this project we did not modified the Grammar}\\

\begin{lstlisting}[frame=single]
Program  = "LOGO" Identifier { Subroutine } { Statement } "END"

Subroutine = "TO" Identifier { Parameter } { Statement } "END"

Statement  = "CS" | "PD" | "PU" | "HT" | "ST" 
    | "FD" NExpr | "BK" NExpr | "LT" NExpr | "RT" NExpr 
    | "WAIT" NExpr 
    | "REPEAT" NExpr "[" { Statement } "]"
    | "IF" BExpr "[" { Statement } "]"
    | "IFELSE" BExpr "[" { Statement } "]" "[" { Statement } "]"
    | Identifier { NExpr } 
      
NExpr    = NTerm { ( "+" | "-" )  NTerm }

NTerm    = NFactor { ( "*" | "/" ) NFactor }

NFactor  = 
    "-" ( Number | REPCOUNT | Parameter | "(" NExpr ")" ) | 
	 Number | REPCOUNT | Parameter | "(" NExpr ")" 

BExpr      = BTerm { "OR" BTerm }

BTerm      = BFactor { "AND" BFactor }

BFactor    = "TRUE" | "FALSE" | "NOT" "(" BExpr ")" 
| NExpr ( "==" | "!=" | "<" | ">" | "<=" | ">=" )  NExpr 

Comments start with "#" with scope until the newline

Numbers are real numbers

Identifiers start with a letter followed by letters or digits

Parameters are ":" followed by Identifier

Identifiers, parameters, keywords in uppercase only
\end{lstlisting}

\chapter{Solution}

First, we added syntax checking for LOGO, without actually implementing an interpreter. This was fairly straight-forward after we got used to javaCC's syntax.

After this, functionality of an interpreter was added step-by-step, starting at the most basic statements and building upon them.

There were some problems, like ensuring that the output is always proper Java syntax.
At first, REPCOUNT didn't work with nested REPEAT statements. We then decided to implement REPCOUNT as a stack; with each deeper level of REPEAT nesting, another element is pushed to that stack, and popped on leaving the loop.

We also experienced problems in getting the turtle to display correctly. Originally, the turtle would only be drawn, never erased - This lead to the turtle being left over after every single instruction.
First, we tried getting around this by drawing the turtle on a separate canvas from the actual drawing. This would have brought various advantages, such as draw speed and leaving the drawing intact when erasing the turtle.

Limitations in java.applet and our knowledge of java lead to this idea being shelved; instead, we found that the method pd() in LogoPrimitives.java always sets the draw color to black, which made erasing the turtle impossible.
After fixing this, the turtle was erased properly, but since it's merely erased by drawing over it in white, it will damage drawings under the turtle.

\chapter{Test}
We test the translator with all the given logo examples and some more specific tests.

That test ensure that REPCOUNT still works and can be complied even with more nested REPEAT statements.

\begin{lstlisting}[frame=single]
# Test 
LOGO TEST

  ST
  REPEAT 5 [
  FD 5
    RT REPCOUNT
    WAIT 20
	REPEAT 5 [
		FD 5
		RT REPCOUNT
		WAIT 20
		REPEAT 5 [
			REPEAT 5 [
				FD 5
				RT REPCOUNT 
				WAIT 20
			]
		]
	]
  ]

END
\end{lstlisting}

\chapter{Limitations}




\end{document}

